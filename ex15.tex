\documentclass[a4paper,10pt]{article} %twocolumn
\usepackage[utf8]{inputenc} % letras acentuadas
\usepackage[portuguese]{babel} % tradução de títulos
\usepackage{algorithm} % ambiente para índice de algoritmos
\usepackage{algpseudocode} % fonte e estilo do algoritmo
\usepackage{tikz} % circuitos e automata
%[noend]

\usetikzlibrary{automata,positioning}

\floatname{algorithm}{Algoritmo} % tradução da palavra algorítimo no ambiente de índice

\title{O algoritmo de conversão de AFND para ER}
\author{Eduardo Couto Dinarte}

\begin{document}

\maketitle

\begin{abstract}

    Este artigo consiste na apresenta\c{c}\~{a}o e explica\c{c}\~{a}o de um algoritmo para converter um aut\^{o}mato finito n\~{a}o determin\'{i}stico num automato finito deterministico e, por fim, converter este numa expressao regular. O metodo consiste em apresentar a teoria com imagens dos tres estados da conversao seguida de um exemplo pratico. O objetivo deste texto e fixar o conteudo de conversao de automatos e familiarizar os autores com a producao de artigos cientificos utilizando a linguagem Latex.

\end{abstract}


\section{Introdução ao Automato Finito Não-Determinístico}

\begin{tikzpicture}[shorten >=1pt,node distance=2cm,on grid,auto] 
   \node[state,initial] (q_0)   {$q_0$}; 
   \node[state] (q_1) [above right=of q_0] {$q_1$}; 
   \node[state] (q_2) [below right=of q_0] {$q_2$}; 
   \node[state,accepting](q_3) [below right=of q_1] {$q_3$};
    \path[->] 
    (q_0) edge  node {0} (q_1)
          edge  node [swap] {1} (q_2)
    (q_1) edge  node  {1} (q_3)
          edge [loop above] node {0} ()
    (q_2) edge  node [swap] {0} (q_3) 
          edge [loop below] node {1} ();
\end{tikzpicture}


O algoritmo \textit{Minimax} trabalha percorrendo uma árvore

\section{O Jogo \textit{Connect-4}}

O jogo \textit{Connect-4} (C4) consiste em...

\section{Implementação}

Para conseguir blablabla



O algoritmo \textit{Minimax} segue abaixo:

\begin{algorithm}
\caption{Algoritmo Minimax}\label{alg:minimax}
\begin{algorithmic}[1]
\Function{minimax}{estado}\Comment{retorna uma ação}
\State \textbf{Entradas}: estado é a configuração atual do jogo
\State $v\gets \mathrm{maxvalor}{(estado)}$
\State \textbf{returna} a ação $a$ em sucessores(estado) cujo valor é $v$ %\Comment{comentario}
% \While{$r\not=0$}\Comment{We have the answer if r is 0}
% \State $a\gets b$
% \State $b\gets r$
% \State $r\gets a\bmod b$
% \EndWhile\label{euclidendwhile}
\EndFunction
\Function{maxvalor}{estado}\Comment{retorna o valor estático}
\If{fim(estado)}
   \State \textbf{retorna} estatico(estado)
\EndIf
\State $v \gets -\infty$
\For{todas ações $a$ nos sucessores(estado)}
    \State $v \gets \max{(v, \mathrm{minvalor}(a))}$
\EndFor
\State \textbf{retorna} $v$
\EndFunction
\Function{minvalor}{estado}\Comment{retorna o valor estático}
\If{fim(estado)}
   \State \textbf{retorna} estatico(estado)
\EndIf
\State $v \gets \infty$
\For{todas ações $a$ nos sucessores(estado)}
    \State $v \gets \min{(v, \mathrm{maxvalor}(a))}$
\EndFor
\State \textbf{retorna} $v$
\EndFunction
\end{algorithmic}
\end{algorithm}


\end{document}
